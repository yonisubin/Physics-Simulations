\documentclass{article}
\author{Yoni Subin}
\usepackage{array}
\usepackage{geometry}
\geometry{a4paper,left=10mm,right=10mm,top=5mm,bottom=5mm}
\begin{document}
    \title{Physics Simulations in Python}
    \maketitle
    \section*{Harmonic Piston with Gas}
    This simulation shows that temperature is really just a form of kinetic energy.
    The gas particals float around at random within a piston and cary some small
    amount of kinetic energy. Each time they collide with the piston,
    there is an elastic collision and the partical passes a small bit of momentum
    to the piston according to the equation 
    $m_1v_1+m_2v_2=m_{1_f}v_{1_f}+m_{2_f}v_{2_f}$. At the end of the simulation,
    we show a graph with the location of the piston as well as the velocity 
    as a function of time, as well as a graph of the kinetic and potential 
    energies and show that their sum is constant due to conservation of enerrgy.
    Note how the graph of the velocity of the piston is not continuous due to 
    the collisions.
    \section*{Diatomic Chain}
    This simulation shows a 1 dimentional chain of alternating types of atoms and
    their interations with each other. When even one atom is out of equilibrium 
    this causes an interation in all of the chain passing along its disturbance.
    All the atoms are approximatly a harmonic relation so we can discribe them as 
    masses in a chain connected by springs. In this simulation, we have assumed the 
    "spring" (which is really just electrical forces) are the same between every atom. 
    Here we see that the forces on mass $n$ are given by the equation 
    $F_n=m_n\ddot{x_n}=\kappa\left(x_{n-1}-\delta x_n\right)-\kappa\left(\delta x_n - x_{n+1}\right)$
    where $\kappa$ is the spring constant and $\delta x_n$ is how far mass $n$ is
    from its equilibrium point.
    \section*{Si Crystal}
    The layout of a silicon crystal can be difficult to imagine. It is similar to an FCC
    lattice, but not quite. I have plotted all of the points in the unit 
    lattace and then shown how they layer one on top of the next to form the crystal.
    \section*{Double Pendulum}
    A double pendulum is a similar system to a regular harmonic occilator, but with an additional
    pendulum connected to the bottom of the first. This increases the complexity and makes it near
    impossible to solve through Newtonian force diagrams. Instead, I analyzied it using the Lagrangian.
    Applying the Euler-Lagrange equation of 
    $\frac{d}{dt}\left(\frac{d\mathcal{L}}{d\dot{\theta}_i}\right)-\frac{d\mathcal{L}}{d\theta_i}$
    I calculated an angular acceleration of:
    $$
    \alpha_1 =\frac{-g(2m_1 + m_2) \sin(\theta_1) - m_2 g \sin(\theta_1 - 2\theta_2) - 2 \sin(\theta_1 - \theta_2) m_2 (\dot{\theta}_2^2 l_2 + \dot{\theta}_1^2 l_1 \cos(\theta_1 - \theta_2))}{l_1 (2m_1 + m_2 - m_2 \cos(2\theta_1 - 2\theta_2))}
    $$$$
    \alpha_2 = \frac{2 \sin(\theta_1 - \theta_2) (\dot{\theta}_1^2 l_1 (m_1 + m_2) + g (m_1 + m_2) \cos(\theta_1) + \dot{\theta}_2^2 l_2 m_2 \cos(\theta_1 - \theta_2))}{l_2 (2m_1 + m_2 - m_2 \cos(2\theta_1 - 2\theta_2))}
    $$
\end{document}